\documentclass{article}

\usepackage[spanish]{babel}
\usepackage{amsmath}
\usepackage[utf8]{inputenc}

\title{Método Simplex}
\author{Karen Lozada}


\begin{document}

\maketitle

\section{Introducción}
\label{sec:introduccion}

El método simplex es un algoritmo para resolver problemas de programación lineal. Fué inventado por George Dantzig en el año 1947.

\section{Ejemplo}
\label{sec:ejemplo}

Ilustraremos la aplicacíon del método simplex con un ejemplo.

\begin{equation*}
  \begin{aligned}
    \text{Maximizar} \quad & 2x_1 +x_2
    
    \text{sujeto a} \quad &
    \begin      {aligned}
      x_1-x_2 &\leq 2\\
      -2x_1+x_2 &\leq 2\\
      3x_1+4x_2 &\leq 12\\
      x_1+x_2 &\geq 1\\
      x_1,x_2 &\geq 0
    \end{aligned}
  \end{aligned}
\end{equation*}

Como una de nuestras desigualdades está volteada con el simbolo $\geq$ entonces multiplicamos ambos lados de la desigualdad por $-1$ obteniendo así la forma estándar:

\begin{equation*}
  \begin{aligned}
    \text{Maximizar} \quad & 2x_1 +x_2
    
    \text{sujeto a} \quad &
    \begin      {aligned}
      x_1-x_2 &\leq 2\\
      -2x_1+x_2 &\leq 2\\
      3x_1+4x_2 &\leq 12\\
     -x_1-x_2 &\leq 1\\
      x_1,x_2 &\geq 0
    \end{aligned}
  \end{aligned}
\end{equation*}

Para obtener la forma simplex, añadimos una variable de holgura por cada desigualdad.

\begin{equation*}
  \begin{aligned}
    \text{Maximizar} \quad & 2x_1 +x_2
    
    \text{sujeto a} \quad &
    \begin      {aligned}
      x_1-x_2+x_3 &\leq 2\\                          
      -2x_1+x_2+x_4 &\leq 2\\
      3x_1+4x_2+x_5 &\leq 12\\
     -x_1-x_2+x_6 &\leq 1\\
      x_1,x_2,x_3,x_4,x_5,x_6 &\geq 0
      \end{aligned}
  \end{aligned}
\end{equation*}

A continuacion obtenemos un \emph{tablero simplex} despejando las variables de holgura:

\begin{equation*}
  \begin{aligned}
      x_3 &=2-x_1+x_2\\                          
      x_4 &=2+2x_1-x_2\\
      x_5 &=12-3x_1-4x_2\\
     x_6 &=-1+x_1+x_2\\
     \hline
     z &=\phantom{-1}+2x_1+x_2
    \end{aligned}
\end{equation*}


\end{document}
