\documentclass{article}

\usepackage[utf8]{inputenc}
\usepackage{amsmath}
\usepackage[spanish]{babel}
\title{Apuntes de Programación Lineal}
\author{Ana karen}
\setlength{\parindent}{0cm}




\begin{document}
\maketitle
\tableofcontents
\section{Forma Estándar}
\label{sec:forma-estandar-y}

La forma estándar de un problema de programación lineal es:

Dados una matriz $A$ y vectores $b,c$, maximizar $c^Tx$ sujeto a $Ax\leq b$.
\subsection{Ejemplo}
Resolver el problema en su forma estándar $Ax\leq b$, determinando A,b y c.

Maximizar $x+y$, sujeto a: $X\geq 0$, $1\leq y\leq 3$, $2x+y\leq 7$.
Resolviendo el problema en forma estándar 

\section{Forma Simplex}
\label{sec:forma-simplex}

Para escribir un problema de programación lineal en la forma simplex, se añade una variable de holgura para cada una de las restricciones que se tienen; es decir, si se tienes 3 restricciones se van a añadir 3 variables de holgura; con lo cual, el problema en foma simplex es:

Dados una matriz $A$ y vectores $b,c$,maximizar $c^Tx$ sujeto a $Ax=b$.\\


\begin{tabular}{|c|c|c|}
  \hline
  & A & B \\
  \hline
  Máquina 1 & 1 & 2 \\
  \hline
  Máquina 2 & 1 & 1 \\
  \hline
\end{tabular}

\begin{equation*}
  \label{eq:1}
  A=
  \begin{pmatrix}
    0 & 1 & 2\\
    3 & 2 & -1
  \end{pmatrix}
  \begin{pmatrix}
    5 & -6 & 2\\
    3 & 2 & 1
   \end{pmatrix}
\end{equation*}




\end{document}

